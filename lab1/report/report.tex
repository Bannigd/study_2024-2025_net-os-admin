\input{preamble.tex}

\title{Отчет по лабораторной работе № 1. \\ [Название]}
\author{Данила Стариков \\ НПИбд-02-22}
\date{2024}

\begin{document}

\maketitle

\tableofcontents
\newpage

\section{Цель работы}
Целью данной работы является приобретение практических навыков установки Rocky Linux на виртуальную машину с помощью инструмента Vagrant.

\newpage
\section{Выполнение работы}
\subsection{Подготовка рабочего каталога}
Подготовка лабораторного стенда проводилась в ОС Linux. Предварительно были установлены:
\begin{itemize}
    \item Vagrant v2.3.4 (\url{https://vagrantup.com});
    \item VirtualBox v7.0.16 (\url{https://www.virtualbox.org}). Стоит отметить, что по умолчанию пакетный менеджер \texttt{dnf} устанавливает версию 7.1.0, в которой присутствует ошибка с настройкой опции графического ускорения, что не позволяет собрать \texttt{box}-файл. Откат к прошлой версии устарняет проблему.
    \item Packer v1.11.2 (\url{https://www.packer.io}).
\end{itemize}

Перед началом работы создан временный каталог \texttt{~/tmp/dastarikov}, в котором проводилась работа. В нем расположили образ ОС Rocky Linux v9.4  (\url{https://www.rockylinux.org/download}), и 2 подкаталога \texttt{packer} и \texttt{vagrant}, где находятся скрипты и конфигурационные файлы, необходимые для сборки \texttt{box}-файла.

Сделали несколько изменений в файлах конфигурации: в файлах \texttt{vagrant/default/01-user.sh} и \texttt{vagrant/default/01-hostname.sh} присвоили переменной \texttt{username} значение \texttt{dastarikov}.

\subsection{Рарзвертывание лабораторного стенда на ОС Linux}
\begin{enumerate}
    \item Перешли в каталог с проектом (в нем также должен располагаться образ OC Rocky Linux):
        \begin{minted}{bash}
        cd /var/tmp/dastarikov/packer
        \end{minted}
    \item В терминале набрали
        \begin{minted}{bash}
        makе help
        \end{minted}
    \item Для формирования \texttt{box}\-файла с дистрибутивом Rocky Linux для VirtualBox в терминале набрали
        \begin{minted}{bash}
        make init
        make box
        \end{minted}
    \item Далее начнинается процесс скачивания, распаковки и установки драйверов VirtualBox и дистрибутива на виртуальную машину. После завершения в каталоге \texttt{packer} сформировался образ \texttt{vagrant\-virtualbox\-rocky\-9\-x86\_64.box}
    \item Скопировали файл \texttt{vagrant\-virtualbox\-rocky\-9\-x86\_64.box} в рабочий каталог в подкаталог \texttt{vagrant}.
    \item Для регистрации образа виртуальной машины в Vagrant в терминале в каталоге \texttt{vagrant} набрали
        \begin{minted}{bash}
        make addbox
        \end{minted}
    \item Запустили виртуальную машину Server, введя
        \begin{minted}{bash}
        make server-up
        \end{minted}
    \item Запустили виртуальную машину Client, введя
        \begin{minted}{bash}
        make client-up
        \end{minted}
    \item Подключились к серверу из консоли:
        \begin{minted}{bash}
        vagrant ssh server
        \end{minted}
    \item Ввели пароль \texttt{vagrant}.
    \item Перешли к пользователю dastarikov
        \begin{minted}{bash}
        su - dastarikov
        \end{minted}
    \item Отлогинились.
    \item Проделали тоже самое для клиента.
    \item Выключили обе виртуальные машины:
        \begin{minted}{bash}
        make server-halt
        make client-halt
        \end{minted}
    \item Убедились, что запуск обеих виртуальных машин прошёл успешно, залогинились под пользователем \texttt{vagrant} с паролем \texttt{vagrant} в графическом окружении.
\end{enumerate}

\subsection{Внесение изменений в настройки внутреннего окружения виртуальной машины}
\begin{enumerate}
    \item Для отработки созданных скриптов во время загрузки виртуальных машин убедились, что в конфигурационном файле Vagrantfile до строк с конфигурацией сервера имеется следующая запись:
        \begin{minted}{bash}
        # Common configuration
        config.vm.provision "common user",
        type: "shell",
        preserve_order: true,
        path: "provision/default/01-user.sh"
        config.vm.provision "common hostname",
        type: "shell",
        preserve_order: true,
        run: "always",
        path: "provision/default/01-hostname.sh"
        \end{minted}
    \item Зафиксировали внесённые изменения для внутренних настроек виртуальных машин, введя в терминале:
        \begin{minted}{bash}
        make server-provision
        \end{minted}
    Затем
        \begin{minted}{bash}
        make client-provision
        \end{minted}
    \item Залогинились на сервере и клиенте под созданным пользователем. Убедитесь, что в терминале приглашение отображается в виде dastarikov@server.dastarikov.net на сервере и в виде dastarikov@client.dastarikov.net на клиенте.
    \item Выключили виртуальные машины.
\end{enumerate}
% Пример вставки изображения.
% \begin{center}
%     \centering
%     \includegraphics[width=\textwidth]{../images/test.png}
%     \captionof{figure}{Подпись к графику.}
%     \label{img:test}
% \end{center}
%

\section{Выводы}

\end{document}
