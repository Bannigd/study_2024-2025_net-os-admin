\documentclass{beamer}
\usepackage[english, russian]{babel}
\usepackage[T2A]{fontenc}
\usepackage[utf8]{inputenc}
\usepackage{indentfirst}
\usepackage{amsmath, amsfonts, amssymb, amsthm, mathtools}
\usepackage[export]{adjustbox}
\usepackage{graphicx} 
\graphicspath{ {./images/} }

\usepackage{subcaption}
\usepackage{verbatim}

\usepackage{minted}{\setlength{\parskip}{0pt}}

\usepackage{hyperref}

\hypersetup{
    colorlinks=true,
    linkcolor=blue,
    filecolor=magenta,      
    urlcolor=black,
    pdftitle={Overleaf Example},
    pdfpagemode=FullScreen,
    }


\title{Отчет по лабораторной работе № 2. \\ Настройка DNS-сервера.}
\author{Данила Стариков \\ НПИбд-02-22}
\date{2024}

\begin{document}

\maketitle
\newpage

\tableofcontents

\newpage
\section{Цель работы}
Приобретение практических навыков по установке и конфигурированию DNS- сервера, усвоение принципов работы системы доменных имён.

\newpage
\section{Выполнение работы}
\subsection{Установка DNS-сервера}
\begin{enumerate}
    \item Загрузили вашу операционную систему и перейдите в рабочий каталог с проектом:
        \begin{minted}{bash}
        cd ~/tmp/dastarikov/vagrant
        \end{minted}
    \item Запустили виртуальную машину server:
        \begin{minted}{bash}
        make server-up
        \end{minted}
    \item На виртуальной машине server вошли под созданным  в предыдущей работе пользователем и открыли терминал. Перешли в режим суперпользователя:
        \begin{minted}{bash}
        sudo -i
        \end{minted}
    \item Установили bind и bind-utils:
        \begin{minted}{bash}
        dnf -y install bind bind-utils
        \end{minted}
    \item В качестве упражнения с помощью утилиты dig сделали запрос к DNS-адресу www.yandex.ru:
        \begin{minted}{bash}
        dig www.yandex.ru
        \end{minted}
Проанализируйте в отчёте построчно выведенную на экран информацию.
\end{enumerate}

\subsection{Конфигурирование кэширующего DNS-сервера}
\subsubsection{Конфигурирование кэширующего DNS-сервера при отсутствии фильтрации DNS-запросов маршрутизаторами}
\begin{enumerate}
    \item В отчёте проанализируйте построчно содержание файлов /etc/resolv.conf, /etc/named.conf, /var/named/named.ca, /var/named/named.localhost, /var/named/named.loopback.
    \item Запустили DNS-сервер:
        \begin{minted}{bash}
        systemctl start named
        \end{minted}
    \item Включили запуск DNS-сервера в автозапуск при загрузке системы:
        \begin{minted}{bash}
        systemctl enable named
        \end{minted}
    \item Проанализировали в отчёте отличие в выведенной на экран информации при выполнении команд
        \begin{minted}{bash}
        dig www.yandex.ru
        \end{minted}
    и
        \begin{minted}{bash}
        dig @127.0.0.1 www.yandex.ru
        \end{minted}
    \item Сделали DNS-сервер сервером по умолчанию для хоста server и внутренней виртуальной сети. Для этого требуется изменить настройки сетевого соединения eth0 в NetworkManager, переключив его на работу с внутренней сетью и указав для него в качестве DNS-сервера по умолчанию адрес 127.0.0.1:
        \begin{minted}{bash}
        nmcli connection edit eth0
        remove ipv4.dns
        set ipv4.ignore-auto-dns yes
        set ipv4.dns 127.0.0.1
        save
        quit
        \end{minted}
    \item Сделали тоже самое для соединения System eth0 (если оно активно):
        \begin{minted}{bash}
        nmcli connection edit System\ eth0
        remove ipv4.dns
        set ipv4.ignore-auto-dns yes
        set ipv4.dns 127.0.0.1
        save
        quit
        \end{minted}
    \item Перезапустили NetworkManager:
        \begin{minted}{bash}
        systemctl restart NetworkManager
        \end{minted}
    Проверили наличие изменений в файле /etc/resolv.conf.
    \item Для настройки направления DNS-запросов от всех узлов внутренней сети, включая запросы от узла server, через узел server внесли изменения в файл /etc/named.conf, заменив строку
        \begin{minted}{bash}
        listen-on port 53 { 127.0.0.1; };
        \end{minted}
    на
        \begin{minted}{bash}
        listen-on port 53 { 127.0.0.1; any; };
        \end{minted}
    и строку
        \begin{minted}{bash}
        allow-query { localhost; };
        \end{minted}
    на
        \begin{minted}{bash}
        allow-query { localhost; 192.168.0.0/16; };
        \end{minted}
    \item Внесли изменения в настройки межсетевого экрана узла server, разрешив работу с DNS:
        \begin{minted}{bash}
        firewall-cmd --add-service=dns
        firewall-cmd --add-service=dns --permanent
        \end{minted}
    \item Убедились, что DNS-запросы идут через узел server, который прослушивает порт 53:
        \begin{minted}{bash}
        lsof | grep UDP
        \end{minted}
\end{enumerate}

\subsubsection{Конфигурирование кэширующего DNS-сервера при наличии фильтрации DNS-запросов маршрутизаторами}
В случае возникновения в сети ситуации, когда DNS-запросы от сервера филь- труются сетевым оборудованием, следует добавить перенаправление DNS-запросов на конкретный вышестоящий DNS-сервер. Для этого в конфигурационный файл named.conf в секцию options добавили
        \begin{minted}{bash}
    forwarders { список DNS-серверов };
    forward first;
        \end{minted}
% Текущий список DNS-серверов можно получить, введя на локальном хосте (на котором развёртывается образ виртуальной машины) следующую команду:
Получили список DNS-серверов:
        \begin{minted}{bash}
    cat /etc/resolv.conf
        \end{minted}
% Например, для хостов в дисплейном классе мы получим следующие данные для конфигурационного файла named.conf виртуальной машины server:
%     forwarders { 10.128.0.240; 80.250.174.240; };
%     forward first;
Кроме того, возможно вышестоящий DNS-сервер может не поддерживать технологию DNSSEC, тогда следует в конфигурационном файле named.conf указать следующие настройки:
        \begin{minted}{bash}
    dnssec-enable no;
    dnssec-validation no;
        \end{minted}
\subsection{Конфигурирование первичного DNS-сервера}
\begin{enumerate}
     \item Скопировали шаблон описания DNS-зон named.rfc1912.zones из каталога /etc в каталог /etc/named и переименовали его в user.net (вместо user укажите свой логин):
        \begin{minted}{bash}
        cp /etc/named.rfc1912.zones /etc/named/
        cd /etc/named
        mv /etc/named/named.rfc1912.zones /etc/named/user.net
        \end{minted}
    \item Включили файл описания зоны /etc/named/user.net в конфигурационном файле DNS /etc/named.conf, добавив в нём в конце строку:
        \begin{minted}{bash}
        include "/etc/named/user.net";
        \end{minted}
(вместо user укажите свой логин).
    \item Открыли файл /etc/named/user.net на редактирование и вместо зоны
        \begin{minted}{bash}
        zone "localhost.localdomain" IN {
            type master;
            file "named.localhost";
            allow-update { none; };
        };
        \end{minted}
    прописали свою прямую зону:
        \begin{minted}{bash}
        zone "user.net" IN {
            type master;
            file "master/fz/user.net";
            allow-update { none; };
        };
        \end{minted}
    Далее, вместо зоны
        \begin{minted}{bash}
        zone "1.0.0.127.in-addr.arpa" IN {
            type master;
            file "named.loopback";
            allow-update { none; };
        };
        \end{minted}
    прописали свою обратную зону:
        \begin{minted}{bash}
        zone "1.168.192.in-addr.arpa" IN {
            type master;
            file "master/rz/192.168.1";
            allow-update { none; };
        };
        \end{minted}
    Остальные записи в файле /etc/named/user.net удалили.
    \item В каталоге /var/named создали подкаталоги master/fz и master/rz, в которых будут располагаться файлы прямой и обратной зоны соответственно:
        \begin{minted}{bash}
        cd /var/named
        mkdir -p /var/named/master/fz
        mkdir -p /var/named/master/rz
        \end{minted}
    \item Скопировали шаблон прямой DNS-зоны named.localhost из каталога /var/named в каталог /var/named/master/fz и переименовали его в user.net (вместо user укажите свой логин):
        \begin{minted}{bash}
        cp /var/named/named.localhost /var/named/master/fz/
        cd /var/named/master/fz/
        mv named.localhost user.net
        \end{minted}
    \item Изменили файл /var/named/master/fz/user.net, указав необходимые DNS-записи для прямой зоны. В этом файле DNS-имя сервера @ rname.invalid. заменили на @ server.user.net. (вместо user должен быть указан ваш логин); формат серийного номера ГГГГММДДВВ (ГГГГ — год, ММ — месяц, ДД — день, ВВ — номер ревизии) [1]; адрес в A-записи заменили с 127.0.0.1 на 192.168.1.1; в директиве \$ORIGIN задали текущее имя домена user.net. (вместо user должен быть указан ваш логин), а затем указаны имена и адреса серверов в этом домене в виде A-записей DNS (на данном этапе должен быть прописан сервер с именем ns и адресом 192.168.1.1).
        % При этом внимательно отнеситесь к синтаксису в этом файле, а именно к пробелам и табуляции. В результате должен получиться файл следующего содержания:
        % NOTE: взять из сгенерированного файла
        \begin{minted}{bash}
        $TTL 1D
        @ IN SOA @ server.user.net. (
        2024072700 ; serial
        1D ; refresh
        1H ; retry
        1W ; expire
        3H ) ; minimum
        NS @
        A 192.168.1.1
        $ORIGIN user.net.
        server A 192.168.1.1
        n\end{minted}
        s A 192.168.1.1
    \item Скопировали шаблон обратной DNS-зоны named.loopback из каталога /var/named в каталог /var/named/master/rz и переименовали его в 192.168.1:
        \begin{minted}{bash}
        cp /var/named/named.loopback /var/named/master/rz/
        cd /var/named/master/rz/
        mv named.loopback 192.168.1
        \end{minted}
    \item Изменили файл /var/named/master/rz/192.168.1, указав необходимые DNS-записи для обратной зоны. В этом файле DNS-имя сервера @ rname.invalid. заменили на @ server.user.net. (вместо user должен быть указан ваш логин); формат серийного номера ГГГГММДДВВ (ГГГГ — год, ММ — месяц, ДД — день, ВВ — номер ревизии); адрес в A-записи заменили с 127.0.0.1 на 192.168.1.1; в директиве \$ORIGIN задали название обратной зоны в виде 1.168.192.in-addr.arpa., затем задали PTR-записи (на данном этапе должна быть задана PTR запись, ставящая в соответствие адресу 192.168.1.1 DNS-адрес ns.user.net). В результате должен получиться файл следующего содержания:
        % NOTE: взять из сгенерированного файла
        \begin{minted}{bash}
        $TTL 1D
        @ IN SOA @ server.user.net. (
        2024072700 ; serial
        34 Лабораторная работа № 2
        1D ; refresh
        1H ; retry
        1W ; expire
        3H ) ; minimum
        NS @
        A 192.168.1.1
        PTR server.user.net.
        $ORIGIN 1.168.192.in-addr.arpa.
        1 PTR server.user.net.
        1 PTR ns.user.net.
        \end{minted}
    \item Далее исправили права доступа к файлам в каталогах /etc/named и /var/named, чтобы демон named мог с ними работать:
        \begin{minted}{bash}
        chown -R named:named /etc/named
        chown -R named:named /var/named
        \end{minted}
    \item В системах с запущенным SELinux все процессы и файлы имеют специальные метки безопасности (так называемый «контекст безопасности»), используемые системой для принятия решений по доступу к этим процессам и файлам. После изменения доступа к конфигурационным файлам named восстановили их метки в SELinux:
        \begin{minted}{bash}
        restorecon -vR /etc
        restorecon -vR /var/named
        \end{minted}
    Для проверки состояния переключателей SELinux, относящихся к named, ввели:
        \begin{minted}{bash}
        getsebool -a | grep named
        \end{minted}
    При необходимости дали named разрешение на запись в файлы DNS-зоны:
        \begin{minted}{bash}
        setsebool named_write_master_zones 1
        setsebool -P named_write_master_zones 1
        \end{minted}
    \item Во дополнительном терминале запустили в режиме реального времени расширенный лог системных сообщений, чтобы проверить корректность работы системы:
        \begin{minted}{bash}
        journalctl -x -f
        \end{minted}
    и в первом терминале перезапустили DNS-сервер:
        \begin{minted}{bash}
        systemctl restart named
        \end{minted}
    % NOTE: Если в логе выдаются сообщения об ошибках, то устраните их и повторно перезапустите DNS-сервер.
\end{enumerate}

\subsection{Анализ работы DNS-сервера}
\begin{enumerate}
    \item При помощи утилиты dig получили описание DNS-зоны с сервера ns.user.net (вместо user должен быть указан ваш логин):
        \begin{minted}{bash}
        dig ns.user.net
        \end{minted}
    и проанализировали его.
    \item При помощи утилиты host проанализировали корректность работы DNS-сервера:
        \begin{minted}{bash}
        host -l user.net
        host -a user.net
        host -t A user.net
        host -t PTR 192.168.1.1
        \end{minted}
(вместо user должен быть указан ваш логин).
\end{enumerate}

\subsection{Внесение изменений в настройки внутреннего окружения виртуальной машины}
\begin{enumerate}
    \item На виртуальной машине server перешли в каталог для внесения изменений в настройки внутреннего окружения /vagrant/provision/server/, создайте в нём каталог dns, в который поместили в соответствующие каталоги конфигурационные файлы DNS:
        \begin{minted}{bash}
        cd /vagrant
        mkdir -p /vagrant/provision/server/dns/etc/named
        mkdir -p /vagrant/provision/server/dns/var/named/master/
        cp -R /etc/named.conf /vagrant/provision/server/dns/etc/
        cp -R /etc/named/* /vagrant/provision/server/dns/etc/named/
        cp -R /var/named/master/* /vagrant/provision/server/dns/var/named/master/
        \end{minted}
    \item В каталоге /vagrant/provision/server создали исполняемый файл dns.sh:
        \begin{minted}{bash}
        touch dns.sh
        chmod +x dns.sh
        \end{minted}
        Открыв его на редактирование, прописали в нём следующий скрипт:
        \begin{minted}{bash}
        #!/bin/bash
        echo "Provisioning script $0"
        echo "Install needed packages"
        dnf -y install bind bind-utils
        echo "Copy configuration files"
        cp -R /vagrant/provision/server/dns/etc/* /etc
        cp -R /vagrant/provision/server/dns/var/named/* /var/named
        chown -R named:named /etc/named
        chown -R named:named /var/named
        restorecon -vR /etc
        restorecon -vR /var/named
        echo "Configure firewall"
        firewall-cmd --add-service=dns
        firewall-cmd --add-service=dns --permanent
        echo "Tuning SELinux"
        setsebool named_write_master_zones 1
        setsebool -P named_write_master_zones 1
        echo "Change dns server address"
        nmcli connection edit "System eth0" <<EOF
        remove ipv4.dns
        set ipv4.ignore-auto-dns yes
        set ipv4.dns 127.0.0.1
        save
        quit
        EOF
        systemctl restart NetworkManager
        echo "Start named service"
        systemctl enable named
        systemctl start named
        \end{minted}
    % Этот скрипт, по сути, повторяет произведённые вами действия по установке и настройке DNS-сервера: подставляет в нужные каталоги подготовленные вами конфигурационные файлы; меняет соответствующим образом права доступа, метки безопасности SELinux и правила межсетевого экрана; настраивает сетевое соединение так, чтобы сервер выступал DNS-сервером по умолчанию для узлов внутренней виртуальной сети; запускает DNS-сервер.
    \item Для отработки созданного скрипта во время загрузки виртуальной машины server в конфигурационном файле Vagrantfile добавили в разделе конфигурации для сервера:
        \begin{minted}{hcl}
        server.vm.provision "server dns",
        type: "shell",
        preserve_order: true,
        path: "provision/server/dns.sh"
        \end{minted}
\end{enumerate}

\subsection{Ответы на контрольные вопросы}
\begin{enumerate}
    \item Что такое DNS?
    \item Каково назначение кэширующего DNS-сервера?
    \item Чем отличается прямая DNS-зона от обратной?
    \item В каких каталогах и файлах располагаются настройки DNS-сервера? Кратко охарактеризуйте, за что они отвечают.
    \item Что указывается в файле resolv.conf?
    \item Какие типы записи описания ресурсов есть в DNS и для чего они используются?
    \item Для чего используется домен in-addr.arpa?
    \item Для чего нужен демон named?
    \item В чём заключаются основные функции slave-сервера и master-сервера?
    \item Какие параметры отвечают за время обновления зоны?
    \item Как обеспечить защиту зоны от скачивания и просмотра?
    \item Какая запись RR применяется при создании почтовых серверов?
    \item Как протестировать работу сервера доменных имён?
    \item Как запустить, перезапустить или остановить какую-либо службу в системе?
    \item Как посмотреть отладочную информацию при запуске какого-либо сервиса или службы?
    \item Где храниться отладочная информация по работе системы и служб? Как её посмотреть?
    \item Как посмотреть, какие файлы использует в своей работе тот или иной процесс? Приведите несколько примеров.
    \item Приведите несколько примеров по изменению сетевого соединения при помощи командного интерфейса nmcli.
    \item Что такое SELinux?
    \item Что такое контекст (метка) SELinux?
    \item Как восстановить контекст SELinux после внесения изменений в конфигурационные файлы?
    \item Как создать разрешающие правила политики SELinux из файлов журналов, содержащих сообщения о запрете операций?
    \item Что такое булевый переключатель в SELinux?
    \item Как посмотреть список переключателей SELinux и их состояние?
    \item Как изменить значение переключателя SELinux?
\end{enumerate}

Пример вставки изображения.
\begin{center}
    \centering
    \includegraphics[width=\textwidth]{../images/test.png}
    \captionof{figure}{Подпись к графику.}
    \label{img:test}
\end{center}


\section{Выводы}

\end{document}
